\documentclass[11pt,a4paper,roman]{moderncv}   
\moderncvstyle{casual}
\moderncvcolor{green}        
\usepackage[utf8x]{inputenc}  
\usepackage[russian]{babel}
\usepackage[scale=0.75]{geometry}

\name{Евлогий}{Сутормин}
\title{резюме} 
\address{Химки}{МО}{}
\phone[mobile]{+7~(915)~104~53~71}
\email{evlogij@gmail.com}
\homepage{evlogii.com}

\begin{document}

\makecvtitle

\section{Образование}
\cventry{2006--2010}{Сыктывкар}{Технологический лицей}{сдал ЕГЭ по информатике на 88 баллов}{}{}
\cventry{2010--2012}{Зеленоград}{МИЭТ}{\textit{Факультет микроприборов и технической кибернетики}}{}{} 

\section{Опыт работы}
\cventry{2013--2015}{Монтажник систем учёта тепловой энергии}{}{}{}{Занимался в основном установкой вычислителей типа ВКТ-7. Пару раз устанавливал автоматику систем отопления (типа продвинутых термостатов).}

\section{Языки}
\cvitemwithcomment{Английский}{Pre-Intermediate}{(Легко читаю техническую документацию)}
\cvitemwithcomment{Русский}{Native ;–)}{}

\section{Computer skills}
\cvitem{Языки}{Pascal в школе, С/С++ в вузе. Разного в веб стеке: RoR, JS, Node.js, Electron, jQuery, HTML, CSS. Много копался с ObjC/Swift. Немного трогал Java и Python.}
\cvitem{Инструменты}{Эффективно гуглю. Не боюсь консоли. Активно пользую шорткатами. Знаю про системы контроля версий и багтрекеры.}

\section{}
\cvitem{}{У меня нет опыта коммерческого программирования, но я схватываю на лету, воспринимаю критику и хочу стать крутым.}
\cvitem{}{Немного моего кода можно глянуть на \href{https://github.com/evlogii}{\textcolor{blue}{гитхабе}}.}
\cvitem{}{Не женат. Без вредных привычек. Есть военный билет.}

\end{document}